\documentclass[bookreview]{jss}

%%%%%%%%%%%%%%%%%%%%%%%%%%%%%%
%% declarations for jss.cls %%%%%%%%%%%%%%%%%%%%%%%%%%%%%%%%%%%%%%%%%%
%%%%%%%%%%%%%%%%%%%%%%%%%%%%%%

%% reviewer
\Reviewer{Christopher J. Lortie\\York University and NCEAS}
\Plainreviewer{Christopher Lortie}

%% about the book
\Booktitle{Applied Time Series Analysis With R. Second Edition.}
\Bookauthor{Wayne A. Woodward, Henry L. Gray, and Alan C. Elliott}
\Publisher{CRC Press}
\Pubaddress{USA}
\Pubyear{2017}
\ISBN{9781498734226}
\Pages{522}
\Price{USD 109.95(P)}
\URL{https://www.crcpress.com/Applied-Time-Series-Analysis-with-R-Second-Edition/Woodward-Gray-Elliott/p/book/9781498734226}
%% if different from \Booktitle also set
%% \Plaintitle{Visualizing Categorical Data}
%% \Shorttitle{Visualizing Categorical Data}

%% publication information
%% NOTE: Typically, this can be left commented and will be filled out by the technical editor
%% \Volume{}
%% \Issue{}
%% \Month{}
%% \Year{2018}
%% \Submitdate{2018-03-31}

%% address of (at least one) author
\Address{
  Christopher J. Lortie\\
  York University and NCEAS\\
  Biology\\
  Toronto, Canada, M3J1P3\\
  E-mail: \email{lortie@yorku.ca}\\
  URL: \url{http://www.christopherlortie.info}
}

%% end of declarations %%%%%%%%%%%%%%%%%%%%%%%%%%%%%%%%%%%%%%%%%%%%%%%


\begin{document}

\textbf{Make time for time series statistics} \newline
Many data include time or have longitudinal dimensionalilty. When these data include an index of time, i.e. measures at regular or periodically successive intervals, statistics that use time sequencing in some capacity are appropriate. There are at least two major categories of statistics - 'time series analysis' to examine trends and address potential periodicity in the data and 'time series foresting' that incorporates time into models with the intent of predicting future outcomes. Most major disciplines of data inquiry now use either or both of these categories of statistics to examine patterns and infer causality. 


\textbf{Content} \newline
This is first and foremost a statistics book \citep{Woodward}.


\bibliography{literature}

\end{document}